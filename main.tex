\documentclass[%
%preprint,
 reprint,
superscriptaddress,
%groupedaddress,
%unsortedaddress,
%runinaddress,
%frontmatterverbose,
showpacs,preprintnumbers,
%nofootinbib,
%nobibnotes,
%bibnotes,
 amsmath,amssymb,
 aps,
 prl,
%pra,
%prb,
%rmp,
%prstab,
%prstper,
%floatfix,
]{revtex4-1}
\usepackage{natbib}
\usepackage{graphicx}% Include figure files
\usepackage[caption=false]{subfig}

\usepackage{siunitx}
\usepackage{dcolumn}% Align table columns on decimal point
\usepackage{bm}% bold math
\usepackage{hyperref}% add hypertext capabilities
\usepackage[mathlines]{lineno}% Enable numbering of text and display math
\usepackage{mathtools}
\usepackage[percent]{overpic}
% \linenumbers\relax % Commence numbering lines %WARNING: not work with subfig!

%\usepackage[showframe,%Uncomment any one of the following lines to test
%%scale=0.7, marginratio={1:1, 2:3}, ignoreall,% default settings
%%text={7in,10in},centering,
%%margin=1.5in,
%%total={6.5in,8.75in}, top=1.2in, left=0.9in, includefoot,
%%height=10in,a5paper,hmargin={3cm,0.8in},
%]{geometry}
\DeclareSIUnit\basepair{bp}

\newcommand{\gwlc}[2][\Omega_0; L_0]{G_\text{\tiny WLC}(#2|#1)}
\newcommand{\ghat}[2][\Omega_0; L_0]{\hat{G}_\text{\tiny WLC}(#2|#1)}
\newcommand{\greens}[2][\Omega_0; L]{G(#2|#1)}
\newcommand{\pathd}[1]{\mathcal{D}\left[#1\right]}
\newcommand{\energy}{\mathcal{E}}
\newcommand{\wigD}{\mathcal{D}}
\newcommand{\RR}{\left\langle{}R^2\right\rangle{}}

\begin{document}
%\preprint{APS/123-QED}
\title{Heterogeneity in Nucleosome Spacing Governs Chromatin Elasticity}% Force line breaks with \\
\thanks{A footnote to the article title}%

\author{Bruno Beltran}
\thanks{These authors contributed equally to this work.}%
\affiliation{%
    Biophysics Program, Stanford University, Stanford, California 94305, USA
}%
\author{Deepti Kannan}%
\thanks{These authors contributed equally to this work.}%
\author{Quinn MacPherson}%
\affiliation{%
    Department of Physics, Stanford University, Stanford, California 94305, USA
}%
\author{Andrew J. Spakowitz}%
\email{ajspakow@stanford.edu}%
% \homepage{http://web.stanford.edu/~ajspakow/}%
\affiliation{%
    Chemical Engineering Department, Stanford University, Stanford, California 94305, USA
}%
\affiliation{%
    Department of Materials Science and Engineering, Stanford University Stanford, California 94305, USA
}%
\affiliation{%
    Department of Applied Physics, Stanford University, Stanford, CA 94305
}%
\date{\today}% It is always \today, today,
             %  but any date may be explicitly specified

\begin{abstract}

In vivo, the many proteins that bind DNA introduce heterogenously-spaced
    kinks in the conformation of the DNA double helix.
%I'm not sure if "otherwise rigid" is the right phrase, since DNA can bend and
    %twist even without proteins?
% theta given 0 unwrap : 62.7
Most notably, the winding of DNA around a histone octamer to form a nucleosome
    creates an effective \ang{62.7} kink in the DNA's conformation, a sharp bend that
    is thermally inaccessible to bare DNA\@.
%perhaps we should mention up front that our model is a wormlike chain with
    %aperiodic defects? that way it makes more sense to talk about the
    %competition between geometrical effects and thermal fluctuations.
We present an analytical model that elucidates the effects of this rigid deformation on the
    physical organization of chromosomal DNA\@.
% Our model predicts that periodic rigid kinks, such as nucleosomes with a fixed
%     nucleosome repeat length, can lead to chromatin chains whose elasticities
%     differ by over an order of magnitude for small changes in the repeat length.
% In contrast, we show that adding any heterogeneity to the nucleosome spacing
%     eliminates this sensitivity, allowing us to robustly predict the Kuhn length
%     of chromatin in various important cell types.
%rephrased to implicitly incorporate the homogenous chain results
We find that heterogenously-spaced kinks in a wormlike chain change the
    zero-temperature configuration from a crystalline fiber to a random walk. This geometric effect overpowers thermal fluctuations in the linker DNA\@.
    We predict that the Kuhn length of \textit{in vivo} chromatin is up to four times
    smaller than that of bare DNA due to the geometric constraints induced by
    nucleosome spacing. We also find that the exact change in elasticity depends on
    the average linker length in a given organism.
% In addiction, we show by merely rearranging the positions of intervening
%     nucleosomes, it is possible change the probability that two distal genomic
%     loci loop together by up to 6 order of magnitude.
In addition, our model predicts that the looping probability of two genomic
    loci can change by as much as 6 orders of magnitude depending on the
    particular spacing of the intervening kinks (nucleosomes).
% This suggests that chromatin remodellers could play a much larger role in
%     anchoring together distal loci than previously anticipated.
% Our work fills a critical gap between models that explicitly incorporate
%     nucleosomes, but are computationally impractical to scale, and existing
%     coarse-grained models of chromatin, which ignore the important contributions
%     of nucleosome geometry.
Our results are broadly applicable to any semiflexible polymer with aperiodic
    defects, and the their implications for the flexibility and cyclizability of
    block copolymers are discussed.
\end{abstract}

% PACS, the Physics and Astronomy Classification Scheme.
\pacs{05.20.--y, 05.40.Fb, 36.20.Ey, 87.10.Ca, 87.14.gk, 87.15--v, 87.16.Sr}
%Use showkeys class option if keyword display desired
%\keywords{chromatin \| nucleosome \| kinked WLC}
\maketitle

%ANDY'S ORIGINAL HIGH LEVEL OUTLINE
%1. structure of genome matters (define nucleosome, brief intro on chromatin)
%2. heterogeneity plays a role in structure:
%in vivo DNA has high propensity for random binding that leads to random
%geometric organization;
%nucleosomes and other DNA-binding proteins (such as HU in bacteria) are not always perfectly spaced along the DNA
%3. Past analytical models of DNA have ommitted heterogeneity.
% two categories of analytical models: fixed NRL, or approximate chromatin as
% some effective wormlike chain to study local DNA mechanics (melting,
% kinking, helical WLC, etc.) DNA's mechanical properties will undeniably
% influence the various scales of physical behavior; however, approximating
% nucleosome-bound DNA as a homogenous polymer fails to account for the
% geometrical features that arise from heterogenous kinks imposed by
% nucleosomes or other DNA-binding proteins.
%4. In this paper, we present a model that includes the thermomechanical
%properties of DNA as well as aperiodic binding of nucleosomes. In the
%competition between these two factors, we find that the geometrical
%configuration of the fiber dominates thermal fluctuations in DNA linkers. We demonstrate
%that heterogeneity introduces order-of-magnitude differences in the elasticity
%of the polymer. We also find that these differences are not simply a composite
%average of the corresponding homogenous chains. We apply our model to the
%latest in vivo nucleosome positioning data to show that realistic variability
%in nucleosome spacing significantly affects the ability of chromatin to form
%loops, which is vital to transcriptional regulation

% \section{\label{sec:intro}Introduction}
In the cell, DNA and its associated proteins---collectively, chromatin---are
    carefully organized both spatially and temporally.
This organization is known to play a role in a myriad of biological processes,
    from controlling gene expression~\cite{hubner2013} to facilitating DNA
    damage repair~\cite{hauer2017,stadler2017}, and has motivated a large body
    of work on the statistical mechanics of the chromatin fiber.

The fundamental repeating unit of Eukaryotic chromatin organization is the
    nucleosome, which is formed by 147 basepairs of DNA wrapped around a
    histone octamer~\cite{cutter2015a}.
Adjacent nucleosomes are connected by linker DNA, whose average length varies
    from $<$\SI{10}{\basepair} in fission yeast\cite{givens2012} to over
    \SI{50}{\basepair} in some human cells\cite{schones2008}.

The statistics of bare DNA are known to quantitatively match a wormlike
    chain (WLC) model with a persistence length {$l_p \approx
    \SI{50}{\nano\metre}$}.
While this makes the WLC an excellent model for linker DNA, few analytical
    models have attempted to include the local geometry of nucleosomes connecting
    the linker DNA\@.
Previously, our lab dealt with the special case of constant linker length,
    modeling chromatin as a WLC backbone and nucleosomes as periodic
    kinks~\cite{koslover2013}.
    %are we just not going to gesture towards other analytical models of DNA that
    %coarse-grain out the nucleosomes -- i.e. helical WLC, kinked WLC,
    %etc.? % we should gesture towards the ones that really were about
    %nucleosomes, but not towards the ones that "could be used to coarse grain
    %out nucleosomes" but never explicitly were.
The constant linker length assumption was motivated by evidence
    that linker lengths are preferentially quantized~\cite{widom1992,wang2008a}.
Moreover, models of helical chromatin fibers with fixed linker
    lengths~\cite{wedemann2002} reproduced properties of the well-established ``\SI{30}{\nano\metre} fiber'' structure of \textit{in vitro}
    chromatin.
%since the citations are nucleosome positioning papers, should we explicitly say
    %that the latest measurements of nucleosome positions suggest that
    %nucleosomes are heterogenously spaced along DNA? % wait, I'm confused,
    %that's what I thought I was saying here actually
However, it is becoming increasingly clear that, \textit{in vivo}, nucleosomes
are heterogenously spaced along the DNA backbone~\cite{lai2018,chereji2018,beshnova2014},
    even if some genomic loci are statistically preferred.
Moreover, modern electron microscopy measurements suggest that \textit{in vivo}
    chromatin is largely disordered~\cite{ou2017}, contrary to the
    ``\SI{30}{\nano\metre} fiber'' picture. However, a comprehensive
    understanding of how nucleosome spacing affects this disordered structure is still
    lacking.
%The lack of consensus on how nucleosome spacing affects chromatin structure
    %necessitates analytical models.
%it's not just random right? hasn't shlick done the Oct4 gene with
    %nucleosome positioning data?
% I assume you mean Bascom 2017? yeah but I mean they still use random linker
    % lengths...just like we do
Thus far, detailed computational models of chromatin with random
    linker lengths have suggested that heterogeneity in nucleosome spacing likely plays an
    important role in chromatin
    architecture~\cite{woodcock1993,collepardo-guevara2014,bascom2017a}.  %we
    %should  probably cite tamar schlick's more recent paper here, wait, isn't it
    %already cited right there? is there a more recent one?
    %maybe it's just me, but why cite the older paper if more recent models do
    %include fluctuations? % if I have two citations, I always put the
    %"foundational" one and the "most recent" one. plus it's two different
    %groups, you kinda want one citation per group to independently do a thing.
    %alos, why not have all three citations? personally I feel like...just cite
    %everyone, why not? especially since there's so few peopel to cite
%However, these methods either do not include the effects of linker
%    DNA fluctuations~\cite{woodcock1993}
    %not claiming this rewrite is any better, but I feel like we need a stronger
    %way of explaining the limitations of schlick's approach, other than
    % to be honest, the easiest solution to this is to include her work in the
    % discussion, not the introduction, since, as andy said, we haven't actually
    % used any of her work for inspiration, although that's what's implied here
However, computational constraints limit these simulations to individual
    chromatin chains with prescribed nucleosome
    locations~\cite{collepardo-guevara2014}.
As a result, there is
    a need for analytical models that can systematically
    investigate the effects of realistic linker length heterogeneity on the
    large-scale conformation of chromatin.
    %this next sentence is good, but I'm concerned that the "what's missing"
    %part of the above literature review is sort of masked until this point. I
    %wonder if we can hint that the role that nucleosomes play in the structure
    %of chromatin is still poorly understood earlier on? I feel like that main
    %point gets lost in all the detailed descriptions of past models. I tried
    %tweaking things but if it's not coming across, leaving this comment to let
    %you know what I'm going for.
Without theory to describe the physical principles that organize these more realistic,
    heterogenous fibers, it has become \textit{de facto} standard to treat
    chromatin as an effective WLC with a persistence length that either matches
    that of bare DNA for
    simplicity~\cite{benedetti2017,macphersonInPress,nuebler2018} or is chosen
    to fit the quantity being measured~\cite{sanborn2015,pierro2017}.

    %is it worth introducing the competition between genometry and fluctuations
    %in the intro? not sure how much to recaptiulate from abstract, but this is
    %the main "physics-y" result to highlight. % i feel like it doesn't make any
    %sense to introduce it before the model, since we need to be more percise in
    %the main text. should bring it up in results/discussion though
In this paper, we present the exact analytical theory for a wormlike chain
    with arbitrarily spaced rigid kinks, and use it to investigate the effects
    of nucleosomes on chromatin structure.
We show histone binding can increase chromatin elasticity by orders of
    magnitude, and that the amount of this increase depends only on the average
    linker length for a given cell type.
Furthermore, we show that aperiodic defects due to nucleosomes---or, in
    practice, any protein that binds DNA---can change the
    propensity for two genomic loci to come into contact by up to six
    orders of magnitude.
Our results suggest that nucleosome positioning can affect not only
    the compaction of genetic material, but also the creation of
    functional chromatin loops. %mobility of locus might be a stretch
    %"correctly" is a vague word to describe models. deleted "interactions"
    %because I generally dislike phrases that have A and B when both are not
    %essential to convey the point. feel free to replace organization with
    %interactions if you prefer.
Furthermore, our results provide theoretical guidance for quantitatively
    coarse-graining simulations of large-scale chromatin organization from the
    bottom up~\cite{macphersonInPress}.

\section{\label{sec:model}Model}
% maybe "make rigorous" is too forceful. Maybe we just cut the first sentence,
% since we never refer to beads on a string in intro?
% I think it's important to keep the "beads on a string" phrase somewhere, since
% it's how people in the field refer to this idea, and it's important for people
% who will be skipping our methods to at least take away that one point. maybe
% put it in intro instead?
%First, we make rigorous our ``beads on a string'' model of chromatin.
We model linker DNA as a WLC and nucleosomes as the points where these
    WLC's connect.
At each nucleosome, we impose the constraint that the incoming and outgoing
    linker DNA strands have a fixed relative orientation. This relative
    orientation $\Omega_\text{kink}  = {(\Omega^{(i)}_\text{entry})}^{-1}
    \Omega^{(i)}_\text{exit}$ between the $i$th and $(i+1)$th linkers is
    determined by the nucleosome structure, as shown in
    Figure~\ref{fig:nuc-geo}.

We represent this sequence of WLC's as a space curve $\vec{R}(s)$, $s\in[0,L]$, along
    with a triad of vectors describing the chain's orientation at each location
    $\vec{t_i}$, where $\vec{t_3} \coloneqq \partial_s \vec{R}(s)$.
% thoughts on deleting this?
% Here, $s$ tracks only along the linker DNA, since each nucleosome is merely
    % considered a point on the curve.
We track the bend and twist of our polymer via the Euler vector $\omega_i$
    defined by $\Omega(s) = \partial_s \vec{t_i}(s) = \omega_i(s) \times \vec{t_i}(s)$.

We begin by computing the Green's function of the first linker, which represents
the probability that the polymer of length $L_1$, which begins at
the origin with fixed initial orientation $\Omega_0$, will end at position
$\vec{R}$ with fixed end orientation $\Omega$.
For a twistable WLC with no kinks, the Green's function is given by
\begin{equation}\label{eq:path}
    \gwlc[\Omega_0;L_1]{\vec{R}, \Omega} =\! \int_{0}^{L_1} \!\! \pathd{\Omega(s)}
              \delta(\vec{R} - \!\int_{0}^{L_1} \!\!\vec{t_3} ds)
              e^{-\beta \mathcal{E}},
    % overflows
    % \greens{\vec{R}, \Omega} = \int_{\Omega(s=0)}^{\Omega(s=L)} \pathd{\Omega(s)}
    %           \exp{[-\beta \mathcal{E}]}
    %           \delta\left(\vec{R} - \int_{s=0}^L \vec{t_3} ds\right)
\end{equation}
    where the energy
\begin{equation}\label{eq:energy}
    \beta\energy = \frac{l_p}{2}\int_{0}^{L_1} ds
    (\omega_1^2 + \omega_2 ^2) + \frac{l_t}{2}\int_{0}^{L_1} ds
    {\left(\omega_3 - \tau\right)}^2
\end{equation}
%TODO get citations for DNA structure things
    is quadratic in bend and in twist. {$\tau \approx \SI{10.5}{\basepair}$}
    per turn is the equilibrium twist density of the DNA double helix, and {$l_p
    \approx \SI{50}{\nano\metre}$}, {$l_t \approx \SI{100}{\nano\metre}$} are
    the bend and twist persistence lengths of DNA, respectively.

This equation can be solved analytically in Fourier space
    ($\vec{R} \rightarrow \vec{k}$) by expanding in Wigner D
    functions~\cite{spakowitz2006}, the eigenfunctions of the heat equation on
    $SO(3)$.
\begin{equation}\label{eq:expansion}
    \ghat{\vec{k}, \Omega} = \sum_{l m j}\sum_{l_0 m_0 j_0} \!\! g_{l_0 m_0 j_0}^{lmj}
        \wigD_l^{mj}(\Omega)\wigD_{l_0}^{m_0 j_0 *}(\Omega_0).
\end{equation}
%side note: removed all instances of "simply". it's a personal pet peeve of
%mine. % haha thank you, leftovers from reading too many math books.
In order to account for the kink introduced by the nucleosome, we rotate
    the final orientation of the linker DNA ${\Omega = \Omega_\text{entry}}$ to
    $\Omega_\text{exit}$ by replacing %why the paranthetical 1? seems
    %unnecessary % to match first paragraph of intro
    %my derivation has this as Omega_kink inverse?
    $g_{l_0 m_0 j_0}^{lmj}$ in Eq.~\ref{eq:expansion} with $B_{l_{0}m_{0}j_{0}}^{lmj} =
    \sqrt{\frac{8\pi}{2l+1}} \mathcal{D}_{l}^{jj_{0}}
    \left(\Omega_{kink}\right)g_{l_{0}m_{0}j_{0}}^{lmj}\left(L\right)$.
The coefficients $B_{l_0 m_0 j_0}^{lmj}(\Omega_\text{kink}, L)$ were first computed
    % stole this citation from Lena's "systematic coarse graining" paper
    in~\cite{zhou2003}, and an alternative derivation can be found in the supplement.

\begin{figure}[t]
    \centering
    \subfloat[]{\label{fig:entry-exit}
        \begin{overpic}[width=100pt]{./figures/fig-1a-nucleosome-geometry.png}
            \put(16,-3){\large$\displaystyle\theta$}
            \put(22,59){\large$\displaystyle\Omega_\text{in}$}
            \put(68,18.5){\large$\displaystyle\Omega_\text{out}$}
        \end{overpic}%
    }\hfill{}
    \subfloat[]{\label{fig:linker-effect}
        \begin{overpic}[width=130pt]{./figures/fig-1b-helicity-effect.png}
            \put(18,12){\parbox{1.5cm}{\centering DNA Helicity}}
            \put(-2,47){\large$\displaystyle\phi$}
            \put(-5,62){$\displaystyle\SI{33}{\basepair}$}
            \put(10,75){$\displaystyle-\SI{2}{\basepair}$}
            \put(40,79){$\displaystyle\SI{35}{\basepair}$}
            \put(63,68){$\displaystyle+\SI{2}{\basepair}$}
            \put(75,47){$\displaystyle\SI{37}{\basepair}$}
        \end{overpic}
    }%
    \caption{\protect\subref{fig:entry-exit} The structure of a
        human nucleosome~\cite{wakamori2015} with straight linkers extrapolated
        based on the entry ($\Omega_\text{entry}$) and exit ($\Omega_\text{exit}$)
        orientations of the bound DNA\@.
    The amount of DNA wrapping the nucleosome dictates the spherical angle
        $\theta$.
    \protect\subref{fig:linker-effect} Two adjacent nucleosomes at zero
        temperature.
    The DNA double helix has an intrinsic twist density
        $\tau=\SI{10.5}{\basepair}$.
    If we anchor the location of one nucleosome, the
        binding orientation of the next histone octamer must change so that it
        aligns with the major groove of the double helix.
    This means that as the linker length $l$ connecting two nucleosomes gets
        longer or shorter, the relative orientations of adjacent octamers
        changes by an angle $\phi = 2l/\tau$.
    }\label{fig:nuc-geo}
\end{figure}

DNA is known to partially unwrap from the nucleosome core dynamically \textit{in
    vivo}, whether spontaneously (called nucleosome breathing~\cite{TODO}), due to
    force on the DNA~\cite{TODO}, or through active
    remodelling~\cite{dion2007,kulaeva2007,senavirathne2017}.
X-ray crystallography~\cite{white2001,richmond2003,cutter2015a} and
    %with H1: bednar, zhou; H4 acetlyation: wakamori; whole fiber: wakamori,
    %song2014b, eltsov2018, bilokapic.
    cryo-EM~\cite{bednar2017,bilokapic2018,eltsov2018,wakamori2015,zhou2015}
    measurements of the nucleosome have established that histone-bound DNA is
    well approximated by a deformed B-DNA structure, wrapping the histone
    octamer 1.7 times in a superhelix with radius \SI{4.19}{\nano\metre} and a
    pitch of \SI{2.59}{\nano\metre}~\cite{richmond2003}.
This means that $\Omega_\text{in}$ and $\Omega_\text{out}$ are well defined as a
    function of how many nucleotides are bound to to the histone core (see
    Supplemental Materials for details). %what details were you envisioning?
    % literally an explicit formula for our omega as a function of unwrapping
%even though/While this process is important for allowing DNA-binding proteins to find
%DNA~\cite{polach1995,anderson2000,li2004}, transcript~\cite{kulaeva2007} and replication
%to take place...
%even though it has been directly measured through cryoem in solution that
    % nucleosomes are on average less wrapped than this~\cite{
In what follows, we fix the wrapping level to that found in the crystal
    structure.
Using different values for the wrapping level does not substantially change our
    results (see Supplemental Figure XX).
    %do you think we should do a treatment of unwrapping for SI? maybe just for
    %statistically preferred amounts of unwrapping with linker length fixed?
    %I also feel like this could be a whole other paper, with both unwrapping
    %and linker length heterogeneity + comparisons to ideal chain. we should
    %talk to Andy about how deep to go with this. % yeah sure, my thoughts are
    %that we should have kuhn length plots maybe for one or two more unwrapping
    %levels and just show that the general trends are the same
    %or even for a bunch of different fixed unwrapping levels tbh, there's no
    %reason not to do it, in fact, I'm going to start that scan right now :)
    %i'll run the 7 plausible symmetric unwrapping cases only for now
    %%to be honest, we could probably move the discussion of the whole
    %nucleosome unwrapping thing to the discussion, i think it would fit better
    %there
To compose monomers of the nucleosome chain with perscribed linker lengths, we
    perform an iterated convolution of the green's function in
    equation~\ref{eq:expansion}.
The position and orientation of the $n$th nucleosome exit site is given by
\begin{equation}\label{eq:conv}
    \greens{\vec{R},\Omega} = \greens[L_n]{\cdot} * \cdots{} * \greens[L_1]{\cdot},
\end{equation}
    where $L = \sum L_i$.
In Fourier space, this corresponds to multiplying the matrices $B(L_i)$
    from Eq.~\ref{eq:expansion}.

Moments of the chain can be computed by noticing that
% \begin{equation}\label{eq:moments}
    $\lim_{k\to0} \frac{\partial^n B_{00}^{00}}{\partial k} = i^n \left\langle
    R^n\right\rangle$.
% \end{equation}
% See the Supplement for details.
    %still think we should include equation for the inversion
To calculate the probability of two loci looping, we numerically invert
    the Fourier transform to evaluate $\greens[L]{\vec{R}}$ at $\vec{R} = 0$.
We note that this corresponds to a modified $J$-factor with no orientational
    component.

A key property of our model is that the relative orientation of adjacent
    nucleosomes is not just determined by $\Omega_\text{kink}$ and the thermal
    fluctuations of the linker strand.
As demonstrated in Figure~\ref{fig:linker-effect}, changing the length of the
    linker strand will change the relative orientation of two adjacent
    nucleosomes, even at zero temperature.
%This effect is thanks to the intrinsic twist density of the DNA
%    ($\tau$ in Eq.~\ref{eq:energy}).
Our propagator $G$ takes this property into account implicitly thanks to the
    inclusion of $\tau$ in Eq.~\ref{eq:energy}.
%it feels weird to quote the uniform and exponential distributions without any
    %motivation for doing so. % agreed, it was a test to see if it would sound
    %as weird as I thought it did, lol.
% In what follows, we study chains with constant linker lengths $L_i = C$, then
%     chains with small amounts of heterogeneity ${L_i \stackrel{i.i.d.}\sim
%     U[C - \sigma, C + \sigma]}$, and finally chains with realistic linker
%     heterogeneity $L_i \stackrel{i.i.d.}\sim \operatorname{Exp}(C)$.



\begin{figure*}[th]
    \begin{centering}
    \subfloat[][]{\label{fig:mc-example}
        \includegraphics[width=0.15\textwidth]{./figures/fig-2b-mc-example.png}
    }%
    \subfloat[][]{\label{fig:homo-kuhn-all}
        \includegraphics[width=0.325\textwidth]{./figures/fig-2a-kuhn-all.png}
    }%
    \subfloat[][]{\label{fig:homo-kuhn-zoom}
        \includegraphics[width=0.5\textwidth]{./figures/fig-2c-kuhn-zoom.png}
    }%
    \end{centering}
    \caption{\protect\subref{fig:mc-example} Zero-temperature structure vs. Monte Carlo simulation
    snapshot of nucleosome chain with \SI{38}{\basepair} linkers.
    \protect\subref{fig:homo-kuhn-all} The Kuhn lengths of homogenous chains are
    \SI{10.5}{\basepair} periodic in linker length, with each linker length
    representing a distinct, discretized helical WLC\@. In the limit of long
    linkers, the Kuhn length eventually approaches that of bare DNA,
    \SI{100}{\nano\metre}. \protect\subref{fig:homo-kuhn-zoom} Compact, superhelical
    structures afford more flexibility than non-compact fibres, which have
    higher Kuhn lengths. Thus, the overall structure of chromatin is extremely
    sensitive to changes in the nucleosome repeat length.}\label{fig:homo-kuhn}
\end{figure*}

\section{\label{sec:model}Results}
\subsection{\label{sec:homo-kuhn}Homogenous Nucleosome Spacing}

%TODO make shorter
We first consider the illustrative case of a homogenous chromatin chain, where
    there is a constant linker length separating adjacent nucleosomes.
Per Chasles's theorem, at zero temperature the nucleosomes in a homogenous chain
    form a superhelix (hereafter, the ``nucleosomal superhelix''\footnote{%
        This is distinct from the DNA superhelix that wraps each the nucleosome,
        which is in turn distinct from the intrinsic double helix of the
        underlying DNA itself.})
    whose compactness is determined by the angle between adjacent linkers.
Because the angle between linkers depends on the linker length (again, see
    Figure~\ref{fig:linker-effect}), the shape of the nucleosomal helices formed by
    the homogenous chains depend on the linker length, as shown in
    Figure~\ref{fig:homo-kuhn-zoom}.

We can use Eq.~\ref{eq:moments} to compute the mean squared displacement, $\RR$,
    of the homogenous chain as function of its linker length.
%TODO check if equation looks pretty
This allows us to extract the elasticity of homogenous chromatin via its Kuhn
    length, $b = \lim_{N\to\infty} \frac{\RR}{\sum_{i=0}^N L_i}$, as shown in
    Figure~\ref{fig:homo-kuhn-all}.
We note that due to Chasles's theorem, this calculation corresponds exactly to
    the Kuhn length of the approriate discrete helical wormlike chain, as first
    done in~\cite{yamakawa1976}.
The effects of dsDNA's \SI{10.5}{\basepair} helicity are evident in the
    periodicity of the resulting Kuhn length of chromatin as shown in
    Figures~\ref{fig:homo-kuhn-all}~and~\ref{fig:homo-kuhn-zoom}.

At short length scales, the $\RR$ behaves like a wormlike chain with reduced
    persistence length, due to the linkers being forced to traverse a helix.
At long length scales, the chain predictably asymptotes to a Gaussian
    chain, again with reduced Kuhn length compared to the underlying linkers.
It is noteworthy that the best fit persistence length at short scales and the
    best fit Kuhn length at long scales are quite different (Supplemental Figure
    XX).
Thus, our full theory is required even for calculations involving this
    simplified, homogenous chromatin chain.

We find that, to first approximation, the Kuhn length is determined by the
    amount of DNA contained per unit of rise along the nucleosomal superhelix,
    as shown in Figure~\ref{fig:homo-kuhn-zoom}.
%TODO Either: change this to explain the kuhn length fixing the length of the
    %chain instead of fixing the lenght of the superhelical chain (per quinn) OR
    %change the figure to have the same amount of superhelical chain
For a given bend and twist persistance length $l_p$ and $l_t$, the flexibility
    of a chain is proportional to the length of the chain.
Because straighter structures---such as the \SI{41}{\basepair} linker chain
    shown in Figure~\ref{fig:homo-kuhn-zoom}---have less chain legnth for a
    given distance along  the nucleosomal superhelix, they stiffer, and
    therefore have longer Kuhn lengths.

Perhaps most notably, the variability between the stiffest and most flexible
    chromatin chains spans over an order of magnitude in Kuhn lengths.
%TODO: color the inaccessible ones
While some of the most compact configurations will sterically inaccesible, this
    demonstrates the drastic effect that nucleosome binding can have on DNA's
    elasticity.

\subsection{\label{sec:homo-kuhn}Heterogenous Nucleosome $\RR$}

We now consider the more relevant scenario, when the nucleosomes are randomly
    spaced along the DNA backbone.
%TODO say what we do first, then why we do it
While certain DNA sequences have higher affinities for
    nucleosomes~\cite{something widom}, in practice the most striking features
    of nucleosome positioning data are nucleosome phasing near promoters, CTCF
    sites, and other sites that sterically inhibit nucleosome
    binding~\cite{widom1992}.
These features are predicted to occur by the simplest model of nucleosome
    positioning, one where nucleosomes are positioned uniformly along the DNA
    backbone, and simply sterically excluded from certain sites.
In terms of linker lengths, this model corresponds approximately to the case
    where linker lengths are independent and exponentially distributed about
    their mean.
Thus, while the exact statistics of \textit{in vivo} linker lengths remain to be
    determined, we use the case of independent, exponentially distributed
    linkers as a reference point to show the effect of linker heterogeneity on
    the statistical mechanics of the chromatin fiber.

In analogy to the case of homogenous chains, we find that the zero temperature
    structure provides good intuition for the behavior of the chromatin chain
    when thermal fluctuations are added.
At zero temperature, a chain with random linker lengths is going to have a
    structure determined purely by the interactions between the nucleosome geometry
    and the DNA's intrinsic twist.
In our model, the angle $\theta$ in Figure~\ref{fig:entry-exit} is fixed by the
    nucleosome geometry, but the angle $\phi$ in Figure~\ref{fig:linker-effect}
    is determined by linker length.
As the heterogeneity in linker length increases, this means that the
    zero temperature heterogenous chain will interpolate between a rigid helix
    (in the case of zero linker length variability) and a modified freely
    rotating chain (in the case when linker variability is high enough to make
    the distribution of $\phi$ approximately uniform over $[0, 2\pi]$).

\begin{figure}[t]
    \centering
    \subfloat[]{\label{fig:exp-kuhns}
        \includegraphics[width=0.45\linewidth]{./figures/fig-4-exp-variance.png}
    }%
    \subfloat[]{\label{fig:box-kuhns}
        \includegraphics[width=0.45\linewidth]{./figures/fig3b-box-47bp-only.png}
    }%
    \caption{\protect\subref{fig:exp-kuhns} We compare the kuhn lengths of realistic chromatin fibers
    against those of the corresponding zero-temperature configurations (red
    triangles) and bare DNA (dashed line). Geometrical kinks and fluctuations
    combine to drastically increase the elasticity of chromatin. Mean linker
    lengths in \textit{S.  cerivisiae}[Chereji], mice embryonic stem cells
    [Voong], and human T cells  [Valouev] are marked.
    \protect\subref{fig:box-kuhns} Kuhn lengths for
    zero-temperature configurations versus fluctuating linkers as a function of
    the amount of variance in uniformly distributed linker lengths with a mean
    of \SI{47}{\basepair}. Adding merely \SI{1}{\basepair} of variability causes the
    zero-temperature configuration to resemble a random walk with ``Kuhn
    length'' (i.e.\ diffusivity) drastically lower than bare DNA's
    \SI{100}{\basepair}, explaining the heterogenous chain's increased
    flexibility.}%\label{fig:het-kuhn}
\end{figure}

As expected, for short linker lengths, $<\SI{20}{\basepair}$, the
    zero-temperature, ``geometric'' model predicts the Kuhn length of the chain
    extremely well (Figure~\ref{fig:exp-kuhns}).
As the average linker length
    increases, the geometric picture becomes increasingly processive since each
    linker is still a rigid rod.
On the other hand, the fluctuating chain's Kuhn length stays markedly below the
    \SI{100}{\nano\metre} Kuhn length of the underlying DNA backbone.
While the Kuhn length approaches \SI{100}{\nano\metre} as the linker length
    increases to infinity, this approach is power-law slow (difference shrinks
    as inverse of the linker lengths, see Supplemental Figure XX).
In fact, for realistic linker lengths, the effect is at least a factor of 2.

While exponentially-distributed linker lengths seem like the most natural
    baseline to take, it is equally natural to ask how much variability in the
    nucleosome spacing is needed to create this effect.
While the answer varies depending on the mean linker length to some degree (see
    Supplemental Figure XX for more examples), we show in
    Figure~\ref{fig:box-kuhns} that as little as one base pair of variability
%TODO this statement is almost certainly too strong
    can lead to the chromatin fiber's structure being governed by the
    nucleosome's geometrical properties.

\subsection{\label{sec:looping}Effects of Nucleosome Spacing on Chromatin
Looping}

Chromatin looping is central to biological processes from transcriptional
    control (e.g.\ via promoter-enhancer search) to DNA damage repair (e.g.\ via
    homologous recombination).
In Figure~\ref{fig:looping}, we evaluate our Green's function at $\vec{R} = 0$
    for an ensemble of chains with linker lengths drawn from an exponential
    distribution with mean \SI{54}{\basepair} (using a uniform distribution
    yields similar results, see Supplemental Figure XX).
This quantity quantifies the propensity for two loci on the same chromosome to
    come into contact with each other as a function of their genomic separation.
As expected, the large length scale behavior matches the Gaussian chain scaling
    of $L^{-3/2}$, with the Kuhn length predicted by our earlier analysis.
In comparison to bare DNA, we find that the looping propensity increases
    drastically along the entire chain, as expected due to intra-superhelical
    contacts induced by the nucleosomes' geometry, as well as due to the
    increased elasticity predicted by the model.
More striking than the increase in the looping propensity is the nearly four
    orders of magnitude over which specific rearrangement of nucleosomes can
    modulate the ability of two loci to loop together.
Even at long length scales, where the chain is approximately Gaussian, it is
    possible to increase or decrease the rate of looping by over an order of
    magnitude.
This has profound implications for the ability of chromatin remodelers to act in
    way that can bring together or keep apart related chromosomal loci based on
    the epigenetic environment.

\begin{figure}[t]
    \centering
    \includegraphics[width=0.95\linewidth]{./figures/fig-5a-exp-looping.png}
    %\includegraphics[width=0.35\linewidth]{./figures/fig-5b-looping-features.png}
    \caption{Each purple line designates an individual chain configuration
    with uniformly random linker lengths between 31 and \SI{51}{\basepair}, with
    the black line representing the average over individual chains. Thus, modulating nucleosome positions can allow the cell to change
    the probability of inter-nucleosomal contacts by up to 6 orders of
    magnitude. The peak in the average curve occurs at around a kilobase, the
    length scale of enhancer-promoter contacts. Although the average,
    heterogenous chain tends towards a Gaussian chain with a shorter
    effective Kuhn length, individual configurations retain memory of rigid kinks
    at intermediate length scales, as indicated by residual peaks in the looping
    probability.}\label{fig:looping}
\end{figure}

Like the Kuhn length, our looping function is largely determined by the
    geometrical properties of the underlying zero-temperature chain.
In Figure~\ref{fig:looping}, two examples of homogenous chains with extremely
    compact or extremely linear geometries are provided to show this.
This suggests that \textit{in vivo}, looping propensity might be largely
    determined by geometrical considerations, at least much more than it is
    determined by the thermal fluctuations in the chromatin fiber.

We notice that the size scale at which this is most true, around
    \SI{1}{\kilo\basepair}, is concommitant with the sizes of loops that must
    %TODO honestly i'm out of brain juice to word this carefully
    often form to construct the transcription initiation complex.
This suggests that the aforementioned phasing of nucleosomes near promoters
    might play a causal role in transcription initiation or repression, and not
    just be a side effect of steric exclusion.

Finally, we note that even though the polymer rapidly approaches the Gaussian
    scaling of $L^\alpha$ where $\alpha \approx -3/2$, the chain is slightly
    non-Gaussian (i.e.\ $\alpha \gtrapprox -3/2$) for size scales even
    approaching that of a full chromosome.
This means that even large-scale, coarse-grained models of chromatin are likely
    to fall short if they rely on Rouse (or WLC) looping probabilities.
This is especially relevant to modeling epigenetic spreading, which requires
    understanding the propensity nucleosomes at various distances have to loop with
    each other.

\section{Discussion}

We present a simplified model of the chromatin fiber that retains only the most
    basic geometrical and physical properties of a chain of nucleosomes
    connected by bare linker DNA.\@
By taking a fully analytical approach, we are able to outline the effects of
    linker length and its heterogeneity on the structure of the chromatin fiber.
Our model excludes various important facets of chromatin's structure.
Most strikingly, we ignore the steric interactions between nucleosomes that is
    well known~\cite{widom1992} to constrain the space of possible linker
    lengths.
In addition, our model ignores the finite size of the nucleosome, treating it as
    a point-like kink in the DNA backbone.

While these choices constrain the ability of our model to make predictions that
    can be tested \textit{in vivo}, our bottom-up approach allows us to
    formalize both new and existing intuition for how linker length affects the
    structure of the chromatin fiber in a unified framework.
% We hope that our framework will facilitate future analytical work that takes
%     these details, and many more, into account.
We expect that our central conclusions---that nucleosomes can modulate
    chromatin's elasticity and facilitate or prevent looping between distal loci
    through their positioning---will only gain nuance as these details are added
    to future analytical works.

While the quantitative values may yet change, we suspect that our estimate of
    chromatin's Kuhn length and looping probabilities will already provide
    guidance to future work relying on coarse-grained models of chromatin.
In particular we show that the common practice~\cite{macphersonInPress,nuebler2018}
    of using a Kuhn length of \SI{100}{\nano\metre}, simply because that is the
    Kuhn length of bare DNA, will likely lead to at least a four-fold
    overestimation of the polymer's stiffness.
In large-scale models of chromatin organization, this could mean the difference
    between heterochromatin and euchromatin segregation recapitulating
    \textit{in vivo} observations, and the chromosomes instead being uniformly
    distributed throughout the nucleus.

Finally, while our model has been designed specifically to address chromatin
    structure, we suspect that the intuition gained here will also apply more
    broadly to block copolymers in solution.
Typical block copolymers have a dihedral ``kink'' at the junctions between
    blocks, and our framework could easily be applied to any particular
    copolymer of interest to relate the elasticity and circularization
    propensity of the polymer to the the block size, analagously to linker
    length.


\section{PRL Guidelines}

Must be submitted to a section, closest fit seems to be
L8--81: Biological and Medical Physics

Other options:
L0--06: Statistical Physics and Thermodynamics
L3--30: Dynamics and Structure of Atoms and Molecules
L6--60: Chemical Physics
but espeically these two:
L8--78: Liquid Crystals and Polymers


3750 words

Include:

Any text in the body of the article;
Any text in a figure caption or table caption;
Any text in a footnote or an endnote

Exclude:

Title;
Author and affiliation listing;
Abstract;
Receipt date, published date, and other publication history;
PACS or Keywords and DOI;\@
References;
Author byline footnotes;
Acknowledgments

Estimating the word equivalent for figures can be simplified by using the aspect
ratio (width / height) of the figure. The estimates would be ((150 / aspect
ratio) + 20 words) for single-column figures, and ((300 / (0.5 * aspect ratio))
+ 40 words) for double column figures.

The word equivalent for displayed math is 16 words per row for single-column
equations. Two-column equations count as 32 words per row.

% \section{\label{sec:level1}First-level heading}

% This sample document demonstrates proper use of REV\TeX~4.1 (and
% \LaTeXe) in mansucripts prepared for submission to APS
% journals. Further information can be found in the REV\TeX~4.1
% documentation included in the distribution or available at
% \url{http://authors.aps.org/revtex4/}.

% When commands are referred to in this example file, they are always
% shown with their required arguments, using normal \TeX{} format. In
% this format, \verb+#1+, \verb+#2+, etc. stand for required
% author-supplied arguments to commands. For example, in
% \verb+\section{#1}+ the \verb+#1+ stands for the title text of the
% author's section heading, and in \verb+\title{#1}+ the \verb+#1+
% stands for the title text of the paper.

% Line breaks in section headings at all levels can be introduced using
% \textbackslash\textbackslash. A blank input line tells \TeX\ that the
% paragraph has ended. Note that top-level section headings are
% automatically uppercased. If a specific letter or word should appear in
% lowercase instead, you must escape it using \verb+\lowercase{#1}+ as
% in the word ``via'' above.

% \subsection{\label{sec:level2}Second-level heading: Formatting}

% This file may be formatted in either the \texttt{preprint} or
% \texttt{reprint} style. \texttt{reprint} format mimics final journal output.
% Either format may be used for submission purposes. \texttt{letter} sized paper should
% be used when submitting to APS journals.

% \subsubsection{Wide text (A level-3 head)}
% The \texttt{widetext} environment will make the text the width of the
% full page, as on page~\pageref{eq:wideeq}. (Note the use the
% \verb+\pageref{#1}+ command to refer to the page number.)
% \paragraph{Note (Fourth-level head is run in)}
% The width-changing commands only take effect in two-column formatting.
% There is no effect if text is in a single column.

% \subsection{\label{sec:citeref}Citations and References}
% A citation in text uses the command \verb+\cite{#1}+ or
% \verb+\onlinecite{#1}+ and refers to an entry in the bibliography.
% An entry in the bibliography is a reference to another document.

% \subsubsection{Citations}
% Because REV\TeX\ uses the \verb+natbib+ package of Patrick Daly,
% the entire repertoire of commands in that package are available for your document;
% see the \verb+natbib+ documentation for further details. Please note that
% REV\TeX\ requires version 8.31a or later of \verb+natbib+.

% \paragraph{Syntax}
% The argument of \verb+\cite+ may be a single \emph{key},
% or may consist of a comma-separated list of keys.
% The citation \emph{key} may contain
% letters, numbers, the dash (-) character, or the period (.) character.
% New with natbib 8.3 is an extension to the syntax that allows for
% a star (*) form and two optional arguments on the citation key itself.
% The syntax of the \verb+\cite+ command is thus (informally stated)
% \begin{quotation}\flushleft\leftskip1em
% \verb+\cite+ \verb+{+ \emph{key} \verb+}+, or\\
% \verb+\cite+ \verb+{+ \emph{optarg+key} \verb+}+, or\\
% \verb+\cite+ \verb+{+ \emph{optarg+key} \verb+,+ \emph{optarg+key}\ldots \verb+}+,
% \end{quotation}\noindent
% where \emph{optarg+key} signifies
% \begin{quotation}\flushleft\leftskip1em
% \emph{key}, or\\
% \texttt{*}\emph{key}, or\\
% \texttt{[}\emph{pre}\texttt{]}\emph{key}, or\\
% \texttt{[}\emph{pre}\texttt{]}\texttt{[}\emph{post}\texttt{]}\emph{key}, or even\\
% \texttt{*}\texttt{[}\emph{pre}\texttt{]}\texttt{[}\emph{post}\texttt{]}\emph{key}.
% \end{quotation}\noindent
% where \emph{pre} and \emph{post} is whatever text you wish to place
% at the beginning and end, respectively, of the bibliographic reference
% (see Ref.~[\onlinecite{witten2001}] and the two under Ref.~[\onlinecite{feyn54}]).
% (Keep in mind that no automatic space or punctuation is applied.)
% It is highly recommended that you put the entire \emph{pre} or \emph{post} portion
% within its own set of braces, for example:
% \verb+\cite+ \verb+{+ \texttt{[} \verb+{+\emph{text}\verb+}+\texttt{]}\emph{key}\verb+}+.
% The extra set of braces will keep \LaTeX\ out of trouble if your \emph{text} contains the comma (,) character.

% The star (*) modifier to the \emph{key} signifies that the reference is to be
% merged with the previous reference into a single bibliographic entry,
% a common idiom in APS and AIP articles (see below, Ref.~[\onlinecite{epr}]).
% When references are merged in this way, they are separated by a semicolon instead of
% the period (full stop) that would otherwise appear.

% \paragraph{Eliding repeated information}
% When a reference is merged, some of its fields may be elided: for example,
% when the author matches that of the previous reference, it is omitted.
% If both author and journal match, both are omitted.
% If the journal matches, but the author does not, the journal is replaced by \emph{ibid.},
% as exemplified by Ref.~[\onlinecite{epr}].
% These rules embody common editorial practice in APS and AIP journals and will only
% be in effect if the markup features of the APS and AIP Bib\TeX\ styles is employed.

% \paragraph{The options of the cite command itself}
% Please note that optional arguments to the \emph{key} change the reference in the bibliography,
% not the citation in the body of the document.
% For the latter, use the optional arguments of the \verb+\cite+ command itself:
% \verb+\cite+ \texttt{*}\allowbreak
% \texttt{[}\emph{pre-cite}\texttt{]}\allowbreak
% \texttt{[}\emph{post-cite}\texttt{]}\allowbreak
% \verb+{+\emph{key-list}\verb+}+.

\bibliography{chromatin}

\end{document}
